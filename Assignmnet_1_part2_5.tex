\documentclass[journal,12pt,twocolumn]{IEEEtran}

\usepackage{setspace}
\usepackage{gensymb}

\singlespacing


\usepackage[cmex10]{amsmath}

\usepackage{amsthm}

\usepackage{mathrsfs}
\usepackage{txfonts}
\usepackage{stfloats}
\usepackage{bm}
\usepackage{cite}
\usepackage{cases}
\usepackage{subfig}

\usepackage{longtable}
\usepackage{multirow}

\usepackage{enumitem}
\usepackage{mathtools}
\usepackage{steinmetz}
\usepackage{tikz}
\usepackage{circuitikz}
\usepackage{verbatim}
\usepackage{tfrupee}
\usepackage[breaklinks=true]{hyperref}
\usepackage{graphicx}
\usepackage{tkz-euclide}

\usetikzlibrary{calc,math}
\usepackage{listings}
    \usepackage{color}                                            %%
    \usepackage{array}                                            %%
    \usepackage{longtable}                                        %%
    \usepackage{calc}                                             %%
    \usepackage{multirow}                                         %%
    \usepackage{hhline}                                           %%
    \usepackage{ifthen}                                           %%
    \usepackage{lscape}     
\usepackage{multicol}
\usepackage{chngcntr}

\DeclareMathOperator*{\Res}{Res}

\renewcommand\thesection{\arabic{section}}
\renewcommand\thesubsection{\thesection.\arabic{subsection}}
\renewcommand\thesubsubsection{\thesubsection.\arabic{subsubsection}}

\renewcommand\thesectiondis{\arabic{section}}
\renewcommand\thesubsectiondis{\thesectiondis.\arabic{subsection}}
\renewcommand\thesubsubsectiondis{\thesubsectiondis.\arabic{subsubsection}}


\hyphenation{op-tical net-works semi-conduc-tor}
\def\inputGnumericTable{}                                 %%

\lstset{
%language=C,
frame=single, 
breaklines=true,
columns=fullflexible
}
\begin{document}


\newtheorem{theorem}{Theorem}[section]
\newtheorem{problem}{Problem}
\newtheorem{proposition}{Proposition}[section]
\newtheorem{lemma}{Lemma}[section]
\newtheorem{corollary}[theorem]{Corollary}
\newtheorem{example}{Example}[section]
\newtheorem{definition}[problem]{Definition}

\newcommand{\BEQA}{\begin{eqnarray}}
\newcommand{\EEQA}{\end{eqnarray}}
\newcommand{\define}{\stackrel{\triangle}{=}}
\bibliographystyle{IEEEtran}
\providecommand{\mbf}{\mathbf}
\providecommand{\pr}[1]{\ensuremath{\Pr\left(#1\right)}}
\providecommand{\qfunc}[1]{\ensuremath{Q\left(#1\right)}}
\providecommand{\sbrak}[1]{\ensuremath{{}\left[#1\right]}}
\providecommand{\lsbrak}[1]{\ensuremath{{}\left[#1\right.}}
\providecommand{\rsbrak}[1]{\ensuremath{{}\left.#1\right]}}
\providecommand{\brak}[1]{\ensuremath{\left(#1\right)}}
\providecommand{\lbrak}[1]{\ensuremath{\left(#1\right.}}
\providecommand{\rbrak}[1]{\ensuremath{\left.#1\right)}}
\providecommand{\cbrak}[1]{\ensuremath{\left\{#1\right\}}}
\providecommand{\lcbrak}[1]{\ensuremath{\left\{#1\right.}}
\providecommand{\rcbrak}[1]{\ensuremath{\left.#1\right\}}}
\theoremstyle{remark}
\newtheorem{rem}{Remark}
\newcommand{\sgn}{\mathop{\mathrm{sgn}}}
\providecommand{\abs}[1]{\left\vert#1\right\vert}
\providecommand{\res}[1]{\Res\displaylimits_{#1}} 
\providecommand{\norm}[1]{\left\lVert#1\right\rVert}
%\providecommand{\norm}[1]{\lVert#1\rVert}
\providecommand{\mtx}[1]{\mathbf{#1}}
\providecommand{\mean}[1]{E\left[ #1 \right]}
\providecommand{\fourier}{\overset{\mathcal{F}}{ \rightleftharpoons}}
%\providecommand{\hilbert}{\overset{\mathcal{H}}{ \rightleftharpoons}}
\providecommand{\system}{\overset{\mathcal{H}}{ \longleftrightarrow}}
    %\newcommand{\solution}[2]{\textbf{Solution:}{#1}}
\newcommand{\solution}{\noindent \textbf{Solution: }}
\newcommand{\cosec}{\,\text{cosec}\,}
\providecommand{\dec}[2]{\ensuremath{\overset{#1}{\underset{#2}{\gtrless}}}}
\newcommand{\myvec}[1]{\ensuremath{\begin{pmatrix}#1\end{pmatrix}}}
\newcommand{\mydet}[1]{\ensuremath{\begin{vmatrix}#1\end{vmatrix}}}
\numberwithin{equation}{subsection}
\makeatletter
\@addtoreset{figure}{problem}
\makeatother
\let\StandardTheFigure\thefigure
\let\vec\mathbf
\renewcommand{\thefigure}{\theproblem}
\def\putbox#1#2#3{\makebox[0in][l]{\makebox[#1][l]{}\raisebox{\baselineskip}[0in][0in]{\raisebox{#2}[0in][0in]{#3}}}}
     \def\rightbox#1{\makebox[0in][r]{#1}}
     \def\centbox#1{\makebox[0in]{#1}}
     \def\topbox#1{\raisebox{-\baselineskip}[0in][0in]{#1}}
     \def\midbox#1{\raisebox{-0.5\baselineskip}[0in][0in]{#1}}
\vspace{3cm}
\title{Assignment 1 (part2)}
\author{MUKUL KUMAR YADAV\\ EE20RESCH14003}
\maketitle
\newpage
\bigskip
\renewcommand{\thefigure}{\theenumi}
\renewcommand{\thetable}{\theenumi}
 Download Latex codes from here
\begin{lstlisting}
https://github.com/EE20RESCH14003/Assignment-1(part2)_5
\end{lstlisting}
Download python code here
\begin{lstlisting}
https://github.com/EE20RESCH14003/Assignment-1(part2)_5
\end{lstlisting}

%

%
\section{\textbf{ Matrix 3.9}}
\textbf{Question No. 73:} 

Find X so that $X\myvec{1&2&3\\4&5&6}=\myvec{-7&-8&-9\\2&4&6}$

\subsection{\textbf{Solution}}

Given that 
\begin{align}\label{eq1}
\Vec{X}\myvec{1&2&3\\4&5&6}=\myvec{-7&-8&-9\\2&4&6}
\end{align}

Equation \eqref{eq1} can be written as 
\begin{align}\label{eq2}
\myvec{1&4\\2&5\\3&6}\Vec{X^T}=\myvec{-7&2\\-8&4\\-9&6}
\end{align}
Equation \eqref{eq2} can be represented as 
\begin{align}\label{eq3}
\vec{A}\vec{x}=\vec{b}
\end{align}

where $\vec{A}=\myvec{1&4\\2&5\\3&6}$ and $\vec{b}=\myvec{-7&2\\-8&4\\-9&6}$

The set of least square solutions of $\vec{A}\vec{x}=\vec{b}$ coincides with the non empty set of solutions of equations $\vec{A}^T\vec{A}\vec{x}=\vec{A}^T\vec{b}.$ 

\begin{align}\label{eq4}
 \hat{x}=(\vec{A}^T\vec{A})^{-1}\vec{A}^T\vec{b} 
\end{align}

\begin{align*}
\vec{A}^T\vec{A}=\myvec{1&2&3\\4&5&6}\myvec{1&4\\2&5\\3&6}
\end{align*}

\begin{align*}
\vec{A}^T\vec{A}=\myvec{14&32\\32&77}
\end{align*}

\begin{align*}
\vec{A}^T\vec{b}=\myvec{1&2&3\\4&5&6}\myvec{-7&2\\-8&4\\-9&6}   
\end{align*}

\begin{align*}
\vec{A}^T\vec{b}=\myvec{-50&28\\-122&64}
\end{align*}

\begin{align*}
(\vec{A}^T\vec{A})^{-1}=\frac{1}{54}\myvec{77&-32\\-32&64} 
\end{align*}

Using equation\eqref{eq4} 
\begin{align*}
\hat{x}=\frac{1}{54}\myvec{77&-32\\-32&14}\myvec{-50&28\\-122&64}\\
=\frac{1}{54}\myvec{54&108\\-108&0}\\
=\myvec{1&2\\-2&0}
\end{align*}
\begin{align*}
\vec{X}=\hat{x}^T=\myvec{1&-2\\2&0}  
\end{align*}

\end{document}